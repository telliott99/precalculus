\documentclass[11pt, oneside]{article} 
\usepackage{geometry}
\geometry{letterpaper} 
\usepackage{graphicx}
	
\usepackage{amssymb}
\usepackage{amsmath}
\usepackage{parskip}
\usepackage{color}
\usepackage{hyperref}

\graphicspath{{/Users/telliott/Github/figures/}}
% \begin{center} \includegraphics [scale=0.4] {gauss3.png} \end{center}

\title{Arithmetic}
\date{}

\begin{document}
\maketitle
\Large

\subsection*{divisibility}

In this chapter we're working with integers or just natural numbers.

Doing division, computing $n/d$, means we want to find $q$ such that:
\[ n = q \cdot d + r \]
and we want $r \ge 0$ but $r < d$.

An important special case is when $r = 0$.  We are concerned to know whether the division is \emph{even} or not:
\[ n = q \cdot d, \ \ \ \ \ \ \frac{n}{d} = q \]

If $r = 0$, then the result of dividing $n/d$ is an integer.

There is a special symbol for that:  if $d$ divides $n$ evenly, we write $d|n$.  But don't mix up the new symbol $|$ with the division symbol seen in the line before (i.e. $n/d$).

Now suppose we have another number $m$ that is also divisible by $d$
\[ m = p \cdot d \]

Then 
\[ m + n = (p + q) \cdot d \]
\[ m - n = (p - q) \cdot d \]

\textcolor{red}{$\bullet$ \ If $d|m$ and $d|n$, $d$ also divides their sum and difference.} 

In general if
\[ a + b = c \]
and $d$ divides any two of them, it divides the third.

In the proof above, we used the principle that additional factors don't matter.

\textcolor{red}{$\bullet$ \ If $d|n$ then $d|an$, where $a$ is any whole number.} 

In particular, powers of $10$ don't matter.  If $d|n$, then $d$ divides any multiple of $n$, such as $100 \cdot n$.

\subsection*{divisibility by $3$ or $9$}

It is very helpful to decide quickly whether a number $n$ is divisible by a given small prime number like $2, 3, 5, 7, 11$.

Clearly, if $n$ is even, its last digit is one of $02468$.

And $n$ is divisible by $5$ if its last digit is one of $05$.

The first trick is that $n$ is divisible by $3$ if its digits add to a multiple of $3$.

Example:

$91$ is not divisible by $3$, but $912$ is.  

When processing digits, we can just subtract and forget any multiples of $3$ along the way.  With $91$, recognize that $9$ is divisible by $3$, so just forget it and move on.  

That leaves $1$, which is not divisible by $3$, or $1 + 2$, which is.

Proof.

Suppose the number is:

\[ abcd = a \cdot 10^3 + b \cdot 10^2 + c \cdot 10^1 + d \]
\[ = a (999 + 1) + b (99 + 1) + c (9 + 1) + d \]

As we proved above, if $x|(y+z)$ and $x|y$ then it must be that $x|z$.  Since $9$ times anything is divisible by $3$, it follows that $3$ must divide $a + b + c + d$ for $abcd$ to be divisible by $3$.

$\square$

A similar thing is true of $9$ except that the digits must add only to $9$.  The proof is the same.

How about $123456789$.  Add the digits:
\[ 1 + 2 + 3 + 4 + 5 + 6 + 7 + 8 + 9 = 9 + 9 + 9 + 9 + 9 \]
Do you see the trick there?

So $123456789$ is evenly divisible by both $3$ and by $9$.

\subsection*{divisibility by $7$}

There is also a test for $7$.  Take the last digit of $n$ away from the number.  Then double it and subtract.  Repeat if necessary.  If you reach a multiple of $7$, then $7$ also divides the larger number.

Example.

Let $n = 3101$.  The last digit is $1$, double it and subtract:
\[ 310 - 2 \cdot 1 = 308 \]
Repeat
\[ 30 - 16 = 14 \]
So yes, $3101$ is divisible by $7$.  We can check pretty easily now that we know it's worth it.
\[ 7 \cdot 400 = 2800 \]
Subtracting, I get $301$.
\[ 7 \cdot 40 = 280 \]
Subtracting, I get $21 = 7 \cdot 3$.  

So $7|3101$ and $3101/7 = 400 + 44 + 3 = 443$.

Proof.

Write
\[ m = n - 21b \]
This is the subtraction step.  We subtract one of $b$ from the last place and twice $b$ from the next to last place.  

We use our three rules from the beginning.  First, the $0$ in the last place goes away, because of what we said about factors of $10$ above.  Ignore the $0$.

Then, we know that $7|(21 \cdot b)$ because $7|21$.  

So now we know that $7|n$ if and only if $7|m$.

$\square$

\subsection*{divisibility by $11$}

Working from \emph{right to left}, add up the digits in odd positions and separately, the digits in even positions.  Subtract the larger from the smaller.  If the difference is a multiple of $11$ (including $0$), the number is divisible by $11$.

For any number where the multiplication did not require you to carry a $1$, this is obvious.

Examples:  $77$, $121$, $198$, $1441$.

If you do have a carry:

\begin{verbatim}
11 x 256 = 2816
6 + 8 = 14
1 + 2 = 3
14 - 3 = 11
\end{verbatim}

See the end for this special one:

\begin{verbatim}
11 x 123456789 = 1358024679
9 + 6 + 2 + 8 + 3 = 28
7 + 4 + 0 + 5 + 1 = 17
28 - 17 = 11
\end{verbatim}

We won't prove this rule.

\subsection*{distributive law}

The the \emph{distributive property} of multiplication over addition says that, for any numbers $a,b,c$:

\[ a \cdot (b+c) = a \cdot b + a \cdot c \]

We used this principle above in trying compute $3101/7$ in our head.  Write:
\[ 3101 = 7 \cdot (400 + \dots) \]

That's the first move.  Then what's left is $3101 - 2800$.  So we need to find $301/7$.  And so on.

I can use this principle to check numbers less than $100$ that end in $3$ for whether they are prime.  To begin with, we have:

\begin{verbatim}
13 23 33 43 53 63 73 83 93
\end{verbatim}

We use the $3$'s trick to remove multiples of $3$:

\begin{verbatim}
13 23 43 53 73 83
\end{verbatim}

Since $7 \cdot 10 = 70$, we know that none of $53, 73, 83$ is divisible by $7$.  I use the times table for the rest:  $3 \cdot 7 = 21$ and $6 \cdot 7 = 42$.  So none of the numbers in our list is divisible by $7$. 

Therefore, they are all prime, as we will see soon.  The smallest number that is divisible by $p$ but not by primes smaller than $p$ is $p^2$.  So we do not need to check for divisibility by $11$.

\subsection*{times table}

I know it's boring, but it is extremely helpful to be proficient with multiplication of small numbers:

\begin{verbatim}
    2  3  4  5  6  7  8   9  10  11  12
 2  4
 3  6  9
 4  8 12 16
 5 10 15 20 25
 6 12 18 24 30 36
 7 14 21 28 35 42 49
 8 16 24 32 40 48 56 64
 9 18 27 36 45 54 63 72  81
10 20 30 40 50 60 70 80  90 100
11 22 33 44 55 66 77 88  99 110 121
12 24 36 48 60 72 84 96 108 120 132 144
\end{verbatim}

And it wouldn't hurt to go up as far as $20$.  The $9$'s trick can be helpful for that row.

There are all kinds of tricks.  Here is one for multiplying $n$ by $11$.  

- write the first digit of $n$

- add the first two digits of $n$ and write them

- continue with each pair of digits to the end

- write the last digit of $n$

The thing that makes it hard is you must "carry" the ones.  Consider $n=123456789$

\begin{verbatim}
 123456789
1234567890
-----------
1357913579
\end{verbatim}

That's \emph{almost} right.

But we need another $1$ in the columns with $9+8, 8+7, 7+6, 6+5$, and that extra $1$ will have to get carried again to the column with $4+3$.  Compare

\begin{verbatim}
1357913579
    1111
1358024679
\end{verbatim}

A bit too easy to make a mistake for my taste.


\end{document}
\documentclass[11pt, oneside]{article} 
\usepackage{geometry}
\geometry{letterpaper} 
\usepackage{graphicx}
	
\usepackage{amssymb}
\usepackage{amsmath}
\usepackage{parskip}
\usepackage{color}
\usepackage{hyperref}

\graphicspath{{/Users/telliott/Github/precalculus/fig/}}
% \begin{center} \includegraphics [scale=0.4] {gauss3.png} \end{center}

\title{Special points}
\date{}

\begin{document}
\maketitle
\Large

We begin with the triangle shown below, picking a point $P$ to be \emph{any point} inside the triangle.  Now draw line segments from each vertex through $P$ and extend them to the opposing side.
\begin{center} \includegraphics [scale=0.4] {Ceva1.png} \end{center}
Since $P$ can be anywhere, the ratio can be anything. Let's call it $x$.

\[ \frac{BX}{XC} = x \]

Line $AX$ divides the whole triangle into two parts.  

\begin{center} \includegraphics [scale=0.4] {Ceva2.png} \end{center}

We know that the area of $\triangle ABX$ is in the same proportion to the area of $\triangle AXC$ as $x$, because they share the same height, while $x$ is the ratio of their bases.  

\[ BX = x \cdot XC \]
\[ A_{ABX} = \frac{1}{2} h \cdot BX = \frac{1}{2} hx \cdot XC = x A_{AXC} \]

Now consider the lower pair of triangles $\triangle BPX$ and $\triangle CPX$

These two also have their areas in the ratio $x$, for the same reason.
\begin{center} \includegraphics [scale=0.4] {Ceva2.png} \end{center}

By subtraction, $\triangle ABP$ and $\triangle ACP$ also have ratio $x$.

So, altogether, we have that
\[ \frac{BX}{XC} = \frac{|ABX|}{|ACX|} = \frac{|BPX|}{|CPX|} = \frac{|ABP|}{|ACP|} = x \]

\subsection*{more sides}

By the same reasoning, if $y=CY/YA$
\[ \frac{|BCP|}{|ABP|} = y \]

and if $z= AZ/ZB$
\[ \frac{|ACP|}{|BCP|} = z \]

Then
\[ xyz = \frac{|ABP|}{|ACP|} \ \frac{|BCP|}{|ABP|} \ \frac{|ACP|}{|BCP|} \]

But all terms cancel, so
\[ xyz = 1 \]
And this is of course true not just for the areas but for the original line segments
\[ xyz = \frac{BX}{XC} \ \frac{CY}{YA} \ \frac{AZ}{ZB} = 1 \]

$\square$

This proof also works in reverse,
\[ xyz = 1 \iff \text{3 lines cross at point P} \]

We will just assume that part.

\subsection*{orthocenter}

So now, for this triangle
\begin{center} \includegraphics [scale=0.25] {ceva4.png} \end{center}

if $\alpha$ is the angle at vertex $A$ and so on, then for example,
\[ BX = AB \ cos \beta \]

and 
\[ \frac{BX}{XC} = \frac{AB \ cos \beta}{AC \ cos \gamma} \]
\[ \frac{CY}{YA} = \frac{BC \ cos \gamma}{AB \ cos \alpha} \]
\[ \frac{AZ}{ZB} = \frac{AC \ cos \alpha}{BC \ cos \beta} \]

When we construct this ratio, all the terms cancel.
\[ \frac{AB cos \beta}{AC \ cos \gamma} \ 
\frac{BC \ cos \gamma}{AB \ cos \alpha} \ 
\frac{AC \ cos \alpha}{BC \ cos \beta} = 1 \]

which means that 
\[ \frac{BX}{XC} \ \frac{CY}{YA} \ \frac{AZ}{ZB} = 1 \]

Therefore, the 3 altitudes all cross at a single point.  That point is the orthocenter, and this is a proof that it exists.

$\square$


\end{document}
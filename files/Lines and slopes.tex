\documentclass[11pt, oneside]{article} 
\usepackage{geometry}
\geometry{letterpaper} 
\usepackage{graphicx}
	
\usepackage{amssymb}
\usepackage{amsmath}
\usepackage{parskip}
\usepackage{color}
\usepackage{hyperref}

\graphicspath{{/Users/telliott/Github/calculus_book/png/}}
% \begin{center} \includegraphics [scale=0.4] {gauss3.png} \end{center}

\title{Analytic geometry}
\date{}

\begin{document}
\maketitle
\Large

It is difficult today to put ourselves in the place of those who tried to reason about mathematics through the ages.  

The Greeks lacked algebra, and although the Romans worked with numbers they did not have decimal notation.  The concept of $0$ came much later (from India), and even in the Middle Ages there was as yet no such thing as the equals sign $=$, which dates from 1557.

\url{https://en.wikipedia.org/wiki/Table_of_mathematical_symbols_by_introduction_date}

The invention of analytic geometry is often ascribed solely to Descartes, but Fermat also had his own version.  There are two fundamental ideas.
\begin{center} \includegraphics [scale=0.45] {coordinates.png} \end{center}

The first is to orient two number lines on a piece of paper, at right angles, and then consider pairs of numbers $(x,y)$ in the 2D plane.  Such pairs or tuples are called points.

Descartes published this idea in 1637.  The presentation would be difficult to recognize as our current system, but the germ is there:  axes where the position of a variable could be marked.  Only the positive numbers would be shown, and the axes not necessarily perpendicular.  As to the proofs, here is wikipedia on the subject:

\begin{quote} \textcolor{blue}{His exposition style was far from clear, the material was not arranged in a systematic manner and he generally only gave indications of proofs, leaving many of the details to the reader.  His attitude toward writing is indicated by statements such as "I did not undertake to say everything," or "It already wearies me to write so much about it," that occur frequently. In conclusion, Descartes justifies his omissions and obscurities with the remark that much was deliberately omitted "in order to give others the pleasure of discovering [it] for themselves."}
\end{quote}

The second idea of analytic geometry is to plot all the points that satisfy some mathematic relationship between $x$ and $y$, for example the parabola $y=x^2$.  

To do this, pick a few values of $x$ and calculate the corresponding values of $y$.  For example:  $(0,0), (\pm 1,1), (\pm 2, 4), \dots$.  Plot these points, and then finally, sketch the graph of the curve, without actually trying to plot \emph{all} of the individual points (of which there is an infinite number).  We make the assumption here that the function being plotted is continuous, so that the sketch of a curve between two points that are close enough together will be fairly smooth and if the $x$-values are close to the plotted $x$, the corresponding $y$-values will not be not too different from the plotted $y$.

\subsection*{point}
A point is simply an ordered pair $(x,y)$ such as $(1,3)$.  Often points have integer components, but they don't have to be.

\subsection*{distance formula}
The $x$- and $y$-axes are perpendicular to one another (a fancy word for that is \emph{orthogonal}).  

Suppose we pick two particular points $(s,t)$ and $(u,v)$, plot them on a graph, and then draw the line that connects them.  Recall Euclid's first two postulates:

$\circ$  A straight line segment can be drawn joining any two points.

$\circ$   Any straight line segment can be extended indefinitely in a straight line.

The distance between the two points is given by the Pythagorean formula, where $\Delta x$ is the change in $x$ and $\Delta y$ is the change in $y$:
\[ d = \sqrt{\Delta x^2 + \Delta y^2} \]
It is often easier to use the squared distance and avoid the square root:
\[ d^2 = \Delta x^2 + \Delta y^2 \]
\[ = (s-u)^2 + (t-v)^2 \]
Switching the order of $(s,t)$ and $(u,v)$ doesn't change the result.

\subsection*{formulas for a line}
 
Now we want to derive an equation that describes (is valid for) all the points or pairs of values $(x,y)$ on this line.  A general approach is to say that the line has some slope $m$, which is defined as $\Delta y$, divided $\Delta x$:
\[ m = \frac{\Delta y}{\Delta x} = \frac{y-y'}{x-x'} \]

This is called the \emph{point-slope equation}.  For any two particular points $(s,t)$ and $(u,v)$ one can plot a line between them.  The slope is
\[ m = \frac{s - u}{t - v} \]

One can write the two points in either order, with the same result since:
\[ \frac{s - u}{t - v} = \frac{u - s}{v - t} \]

Depending on the details, the value of $m$ might be zero, for a horizontal line, where all the values of $y$ are the same (which happens when $s = u$).  Or it might be undefined, for a vertical line, where all the values of $x$ are identical ($t = v$).  

In most cases, however, $m \ne 0$ and $m \in (-\infty, \infty)$.  That is, $m$ is usually non-zero and not infinite.

Except in the case of the vertical line, we can write
\[ y = mx + y_0 \]
for any point $(x,y)$ on a given line, where $y_0$ is the $y$-intercept, the value of $y$ obtained when $x = 0$.

[ The choice of $b$ for the $y$-intercept is the usual notation, but it conflicts with another $b$ that we will see in a minute. ]

$y = mx + y_0$ is the \emph{slope-intercept equation} of the line.

The equation of a line is determined by both the slope and one point on the line, for example the $y$-intercept.  One can draw a whole family of parallel lines with the same slope and different $y$-intercepts.  Here are three lines $y = 2x + y_0$ for $y_0 = \{ 0, 1, 2 \}$.

\begin{center} \includegraphics [scale=0.4] {line_family.png} \end{center}

The value of $x$ corresponding to $y = 0$ is the $x$ intercept
\[ x_0 = -\frac{y_0}{m} \]

The point-slope equation is easily derived from the second one.  Suppose we have $y = mx + y_0$:

Plugging in for specific points $(s,t)$ and $(u,v)$ we have
\[ t = ms + y_0 \]
\[ v = mu + y_0 \]
Subtracting:
\[ v - t = m(u - s) \]
which rearranges to give the desired result.

\subsection*{intersections}
Often one has two lines (or curves) and we want to find the point(s) that lie on both.  We might have
\[ y = 2x - 1 \]
\[ y = -x + 8 \]
Substitute from the second into the first:
\[ 2x - 1 = -x + 8 \]
\[ 3x = 9 \]
\[ x = 3 \]
From the first equation, $y = 5$, and we check that $x = 3, y = 5$ solves the second equation as well.

\subsection*{orthogonality}
If two lines cross each other at right angles we say they are \emph{orthogonal}.  In that case the slopes have a special relationship.  Their product is equal to $-1$.

\begin{center} \includegraphics [scale=0.4] {slopes_ortho.png} \end{center}

Here is a simple proof.  Draw the two lines going through the origin, forming a right angle there.  The first has slope $s$, so it goes through the point $(1,s)$, the second goes through $(1, t)$.  

Recall from the chapter on the Pythagorean theorem that the altitude squared is equal to the product of the two pieces of the base.  Here:
\[ 1^2 = 1 = st \]
These are the lengths, i.e. the absolute values of the slopes.  Thus $ |s| = 1/|t|$.  But   

Clearly the sign of $t$ is negative.  So we arrive at
\[ s \cdot (-t) = 1 \]
\[ m_1 = - \frac{1}{m_2} \]

We'll see a natural easy proof of this once we look at trigonometry.  Here is a hint:

\begin{center} \includegraphics [scale=0.4] {rotation.png} \end{center}

\end{document}
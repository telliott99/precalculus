\documentclass[11pt, oneside]{article} 
\usepackage{geometry}
\geometry{letterpaper} 
\usepackage{graphicx}
	
\usepackage{amssymb}
\usepackage{amsmath}
\usepackage{parskip}
\usepackage{color}
\usepackage{hyperref}

\graphicspath{{/Users/telliott/Github/number_theory/png/}}
% \begin{center} \includegraphics [scale=0.4] {gauss3.png} \end{center}

\title{Integers}
\date{}

\begin{document}
\maketitle
\Large

\subsection*{infinity}
The symbol for infinity is $\infty$.

In the old days, they used to write things like
\[ \frac{1}{0} = \infty,  \ \ \ \ \ \ \frac{1}{\infty} = 0 \]

John Wallis wrote $24/0 = \infty$, in 1656, which is when the $\infty$ symbol was introduced with its current definition.  Even Euler argued that $n/0 = \infty$ when it suited him,

It is claimed that the symbol derives from the Roman symbol for 100 million.  That's interesting.  I never knew any symbols larger than $M$, for one thousand.  And I'm not sure I believe it, but that's what some people say.

According to

\url{https://notevenpast.org/dividing-nothing/}

\begin{quote}
On 21 September 1997, the USS Yorktown battleship was testing “Smart Ship” technologies on the coast of Cape Charles, Virginia. At one point, a crew member entered a set of data that mistakenly included a zero in one field, causing a Windows NT computer program to divide by zero. This generated an error that crashed the computer network, causing failure of the ship’s propulsion system, paralyzing the cruiser for more than a day.
\end{quote}

\subsection*{no division by zero}
There is a fundamental problem when we set up a division problem and $0$ is in the denominator.  What goes wrong when we attempt to divide by zero?
\[ \frac{a}{0} = \ ? \]

Well, what do we mean by an expression such as 
\[ \frac{a}{b} = c \]

By \emph{definition}, we mean that we will try to find $c$ such that
\[ c \cdot b = a \]

For the integers, of course, there is the problem of a possible remainder.  Let us leave that aside for a minute.

Suppose we have $c \cdot b = a$ but then take $b$ to be very small though not 0.  In that case, the number $c$ may get very large.  That's OK.  

We can make $b$ as small as we wish by making $c$ large enough or vice versa.  And we can say that as $b \rightarrow 0$, then $c \rightarrow \infty$.

But we can't say $a/0 = $ some number.

If there were such a number (say $a/0 = \infty$, infinity), then what about 
\[ \frac{b}{0} = \ ??, \ \ \ \ \ \ \frac{c}{0} = \ ??  \]
It would mean that whatever the expression $b/0$ is equal to, when multiplied by zero, we would obtain any number whatsoever.  This makes no sense.

Here is another, perhaps silly, example.
\[ 0 \cdot 1 = 0 \]
\[ 0 \cdot 2 = 0 \]
so
\[ 0 \cdot 1 = 0 \cdot 2 \]
but then
\[ 1 = 2 \]

By definition, we do not allow division by zero.

\subsection*{infinity is not a number}

And we can't answer the question what is $2 \cdot \infty$?  If we allowed multiplication by $\infty$ then the only reasonable answer would be
\[ 2 \cdot \infty = \infty \]
so then also
\[ n \cdot \infty = \infty \]
where $n$ is any number.  But then say
\[ 2 \cdot \infty = 3 \cdot \infty \]
so, cancelling
\[ 2 = 3 \]
This would be a mess.

By definition, \emph{infinity is not a number} and division by $0$ is \emph{undefined}.

\subsection*{limits}
Often people say that calculus is all about limits, and they are certainly where you start in proving the theoretical basis of the field.  

We will keep the discussion of limits and $\epsilon$-$\delta$ formalism to a minimum for the reasons explained in the Introduction.  But let us try to establish an intuitive idea about what we mean when we say "in the limit as $N \rightarrow \infty$".

Above we had that there is no greatest integer.

A corollary of that is the limit
\[ \lim_{n \rightarrow \infty} \frac{(n + 1) - n}{n} = 0 \]
Why?  As $n$ increases without bound, the difference between successive numbers, as a fraction of $n$, tends to zero.

To get an idea about this, first simplify by multiplying by $1/n$ on top and bottom.  Then we have
\[ \lim_{n \rightarrow \infty} \frac{(1 + 1/n - 1)}{1} = \frac{1}{n} \]

We say that $1/n$ \emph{tends} to zero as $n \rightarrow \infty$, and so does $[(n+1)-n]/n$.


\end{document}
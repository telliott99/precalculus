\documentclass[11pt, oneside]{article} 
\usepackage{geometry}
\geometry{letterpaper} 
\usepackage{graphicx}
	
\usepackage{amssymb}
\usepackage{amsmath}
\usepackage{parskip}
\usepackage{color}
\usepackage{hyperref}

\graphicspath{{/Users/telliott/Github/figures/}}
% \begin{center} \includegraphics [scale=0.4] {gauss3.png} \end{center}

\title{Prime factorization}
\date{}

\begin{document}
\maketitle
\Large

%[my-super-duper-separator]

We will prove that every integer has a unique \emph{prime factorization}.  This is also called \emph{the fundamental theorem of arithmetic}.

\[ n = p_1 \cdot p_2 \dots p_k \]

In the list of the prime factors of $n$, a factor may be repeated.

Example:

\[ 12 = 2 \cdot 2 \cdot 3 \]

To compare two factorizations for uniqueness, we suppose they are sorted (say, from smallest to greatest).

More examples for relatively small numbers:
\begin{verbatim}
 39  = 3.13
144  = 2.2.2.2.3.3
210  = 2.3.5.7
2310 = 2.3.5.7.11
\end{verbatim}

Sometimes the factors can be hard to find.

Example:

Let's try $123456789$.  By using the digit addition trick, we can tell that this number is divisible by $9$ (I get $9 + 9 + 9 + 9 + 9$).

At first it seems easy.  I found two factors of $3$, leaving $13717421$.

Then my luck ran out.  The smallest prime factor was too large for me to find by hand.  So I used Python.

\url{https://gist.github.com/telliott99/3043a0d9ddc44f8503c83c848b2f8382}

$3607$ and $3803$ are the two prime factors of that number, which once found, are easily confirmed.  Factoring is a hard problem.

\subsection*{background}

We will prove the unique prime factorization theorem.

But before starting on that, when we say that one integer \emph{evenly divides} another one, written as $a|n$ or $a$ is a factor of $n$, we mean there exists another integer $k$ such that

\[ a \cdot k = n \]
$a$ times $k$ is exactly equal to $n$ with no remainder.

Take care to distinguish $a|b$ ($a$ divides $b$) from $a/b$ ($a$ divided by $b$).

If there is also a number $m$ where $a$ evenly divides $m$, we write $a|m$ and mean that
\[ a \cdot j = m \]

Addition or subtraction of $m + n$ gives
\[ m + n = a \cdot j + a \cdot k = a(k + j) \]
\[ m - n = a \cdot j - a \cdot k = a(k - j) \]

$\bullet$ \ If $a|m$ and $a|n$, $a$ also divides their sum or difference.

We rely on this fact below.

Also, we can manually check all the numbers up to some reasonable lower limit, like $100$.  They all have unique prime factorizations.  Therefore, if there is a number with two such factorizations, it is larger than $100$, and there must be a smallest such number.

\subsection*{abnormal numbers}

Hardy and Wright (\emph{Theory of Numbers}, sect. 2:11) have a proof of prime factorization which I find quite elegant.

\emph{Proof}.

By contradiction.

Hardy:
\begin{quote}Let us call numbers which can be factored into primes in more than one way, \emph{abnormal}, and let $n$ be the smallest abnormal number.\end{quote}

Start by supposing that there are two different factorizations of $n$:
\[ n = p_1 \cdot p_2 \dots p_k \]
and
\[ n = q_1 \cdot q_2 \dots q_j \]
where the $p$'s and $q$'s are all primes.

\subsection*{Different factorizations}

As a preliminary result, consider the possibility that some factor appears in both factorizations, that some $p$ is equal to a $q$.

Let us rearrange if necessary so the shared factor is listed first:  let $p_1 = q_1$ and

\[ n = p_1 \cdot p_2 \dots p_k \]
\[ n = p_1 \cdot q_2 \dots q_j \]

But now, $n/p_1$ ($n$ divided by $p_1$) is abnormal, because it has two different prime factorizations.  

That is impossible, because $n$ is the smallest abnormal number.

Therefore, no $p$ is a $q$ and no $q$ is a $p$.  If there exist abnormal numbers with two factorizations, those factorizations must be completely different.

\subsection*{inequality}

We may take $p_1$ to be the least $p$ and $q_1$ to be the least $q$.  In this part, we establish that $p_1 \cdot q_1 < n$.

Since $n$ is composite, either 

$\circ$ \ $p_1 \cdot p_1 = n$, or

$\circ$ \ $p_1$ times some number larger than $p_1$ is equal to $n$.

In the second case, $p_1 \cdot p_1 < n$.

A similar result holds for $q_1$.

But, since $p_1 \ne q_1$, only one of $p_1$ or $q_1$ at most, can be squared to give $n$.  Either $p_1 \cdot p_1 < n$ or $q_1 \cdot q_1 < n$ or perhaps both are true.

From this it follows that 
\[ p_1 \cdot q_1 < n \] 

\subsection*{the contradiction} 

Let 
\[ N = n - p_1 q_1 \]
We known that $N$ is not abnormal (because $n$ is the smallest abnormal number).

We have that $N > 0$ because of the result of the last section.  We also know that $N < n$, since $p_1$ and $q_1$ are non-zero and we subtracted them from $n$.  So $0 < N < n$.

We're given that $p_1 | n$ and so, from the above equality 
\[ N = n - p_1 q_1 \]
and our preliminary result about what factorization means, it must be that $p_1 | N$. The same is true for $q_1$, namely $q_1 | N$.  

Hence both $p_1$ and $q_1$ appear in the unique factorization of $N$, so $p_1q_1 | N$.

Certainly, $p_1 q_1$ divides itself, so it follows that $p_1 q_1 | n$, using our preliminary result about factorization.

Hence, $q_1| (n/p_1)$.  

But $n/p_1$ is less than $n$ and has the unique prime factorization $p_2 \cdot p_3 \dots p_k$.

Since $q_1$ is not a $p$, this is impossible.  

Hence there cannot be any abnormal numbers.

$\square$

As we said, this is \emph{fundamental theorem of arithmetic}, so it's worth a bit of effort for the proof.

\end{document}
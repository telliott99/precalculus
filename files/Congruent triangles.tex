\documentclass[11pt, oneside]{article} 
\usepackage{geometry}
\geometry{letterpaper} 
\usepackage{graphicx}
	
\usepackage{amssymb}
\usepackage{amsmath}
\usepackage{parskip}
\usepackage{color}
\usepackage{hyperref}

\graphicspath{{/Users/telliott/Github/precalculus/fig/}}

\title{Basic geometry and congruence of triangles}
\date{}

\begin{document}
\maketitle
\Large

\subsection*{congruent triangles}

$\bullet$  Two triangles are \emph{congruent} if and only if they have the same three side lengths.  This is often abbreviated SSS (side-side-side).  

As we'll see, some other equalities are equivalent to (that is, they imply) congruence and SSS equality.

By this definition, a triangle and its mirror image are congruent.  The three triangles shown below are all congruent, even though two are flipped (they are mirror images of the other two).

\begin{center} \includegraphics [scale=0.4] {congruent.png} \end{center}

Having the same three sides means that the shape is the same, and all three angles are the same --- the shapes are superimposable, with the proviso that we allow the shape to be flipped over.

In addition to SSS (side-side-side), there are three other conditions that lead to congruence of two triangles when they are satisfied, namely

$\bullet$  SAS (side-angle-side)

$\bullet$  ASA (angle-side-angle)

$\bullet$  AAS (angle-angle-side)

Some triangles are \emph{similar} but not congruent.

Similarity means that the three angles are the same but the triangles are of different overall sizes.  We might say that they are the same but \emph{scaled} differently.  

We can call this AAA (angle-angle-angle).  For similar triangles, the three corresponding pairs of sides are in the same proportions, but re-scaled by a constant of proportion.

$\bullet$  Two triangles are similar if they have the same three angles. 

Because of the angle sum theorem, if any two angles of a pair of triangles are known to be equal, then the third one must be equal as well.

$\bullet$  Two triangles are similar if they have two angles known to be equal. 

Similar triangles have their sides in the same proportion.  This is known as the AAA similarity theorem.

\begin{center} \includegraphics [scale=0.4] {similar.png} \end{center}

Given any triangle, draw a line parallel to one side, which also joins the other two sides.  The new triangle with that side as its base is similar to the given triangle.  Similarity means that all the angles are equal.

\begin{center} \includegraphics [scale=0.25] {Thales_theorem_1.png} \end{center}

In this example, these ratios are all equal
\[ \frac{AD}{AB} = \frac{AE}{AC} = \frac{DE}{BC}  \]
\[ \frac{AD}{DB} = \frac{AE}{EC} = \frac{DE}{BC - DE}  \]

There are some subtleties to similarity.  For now we will assume the theorem:  that AAA and sides in proportion are both the same as similarity.

\subsection*{constructions}

The way I think about these conditions is to imagine trying to construct a triangle from the given information, and ask whether it is uniquely determined.  Suppose we know ASA.  The situation is thus:

\begin{center} \includegraphics [scale=0.4] {ASA1.png} \end{center}
 
Draw the known side, then using the known angles, start two other sides from the ends of that side.  They must cross at a unique point.  

But... actually, if we start the two lines pointing below the horizontal, there is another solution, the mirror image.  This triangle is also congruent to the one above.
 
\begin{center} \includegraphics [scale=0.4] {ASA2.png} \end{center}

If we know two angles we also know the third, because they must add to 180 degrees.  For this reason, ASA and AAS imply that we have exactly the same information, because we know all three angles and (this part is important) we also know \emph{which} two angles flank the known side.

Alternatively, it is enough to know which angle faces the known side.
 
\subsection*{SAS, ASA, AAS but not ASS}

SAS is very commonly used to prove congruence.  
\begin{center} \includegraphics [scale=0.4] {SAS.png} \end{center}

In this diagram, sides of equal length are indicated by one or more hash marks.  Equal angles are indicated by dots (another common method is to draw an arc with a hash across it).

The other methods for proving congruence use two equal angles and a side.  Two equal angles imply the third angle is also equal (since they add to a half-circle or 180 degrees), so the two triangles are similar.  To prove they are congruent, It is important that the equal sides are flanked by the same angles, or equivalently, are opposite the same angle.

These methods using two angles are referred to as ASA
\begin{center} \includegraphics [scale=0.4] {ASA3.png} \end{center}

 and AAS.
\begin{center} \includegraphics [scale=0.4] {AAS.png} \end{center}

\subsection*{one that doesn't work}

There is one set of three that doesn't work, that is ASS (angle-side-side).

\begin{center} \includegraphics [scale=0.4] {angle_side_side.png} \end{center}

Here we know sides $a$ and $b$ and the angle $\theta$ adjacent to $a$ and facing opposite side $b$.  Imagine $b$ swinging on a hinge at the blue dot.  If $b < a$, there are two points where $b$ can intersect with the side projecting from angle $\theta$.  There is no unique solution, so the triangle is not determined.

If it had been the case that $b > a$, or alternatively that $b$ formed a right angle with the third side, then the triangle \emph{would} be determined.

I think Tony Randall said it best

\url{https://www.youtube.com/watch?v=KEP1acj29-Y}

\end{document}